
%\documentclass[12pt]{report}

\documentclass[12pt]{article}
%\usepackage{natbib}  % used for citations
\usepackage[parfill]{parskip} %used for formatting style of text



\usepackage{graphicx,fancyhdr}
\usepackage[dvipsnames]{xcolor}
\usepackage{amssymb,amsmath}
\usepackage[margin=0.8in]{geometry}
\usepackage{epigraph,fancyvrb,eqparbox}
\usepackage[multiple]{footmisc}
\usepackage{menukeys}
\usepackage{menukeys}
\usepackage{url}
\usepackage[colorlinks = true, linkcolor = blue, urlcolor = blue]{hyperref}
\usepackage{setspace}


\pagestyle{fancyplain}

%\usepackage{hyperref}
%\usepackage{epsf,psfig,graphicx,fancyheadings}
% \textwidth 7in
% \textheight 9in
% \oddsidemargin 0in
% \topmargin -.25in

%-----------------------------------------------
% The following settings are from Dr. Davidian's
% ST810A Handout on Advanced LaTeX Features

%\setlength{\paperheight}{11.0in}
%\setlength{\paperwidth}{8.5in}

%%%%%%%%%%%%%%%%%%%%%%%%%%%%%%%%%%%%%%%%%%%%%%%%%
% For Desktop @ CalPoly (for Postscript)

%\setlength{\oddsidemargin}{0.5in}
%\setlength{\evensidemargin}{0.5in}
%\setlength{\topmargin}{-.5in}

%%%%%%%%%%%%%%%%%%%%%%%%%%%%%%%%%%%%%%%%%%%%%%%%%
% For Laptop @ Calpoly (for Postscript)

% \setlength{\oddsidemargin}{0.in}
% \setlength{\evensidemargin}{0.in}
% \setlength{\topmargin}{0.25in}

%%%%%%%%%%%%%%%%%%%%%%%%%%%%%%%%%%%%%%%%%%%%%%%%%
% For Desktop @ CalPoly (for PDF)

%\setlength{\oddsidemargin}{0.in}
%\setlength{\evensidemargin}{0.in}
%\setlength{\topmargin}{-.5in}
%
%%%%%%%%%%%%%%%%%%%%%%%%%%%%%%%%%%%%%%%%%%%%%%%%%%
%% For Laptop @ Calpoly (for PDF)
%
%% \setlength{\oddsidemargin}{0.in}
%% \setlength{\evensidemargin}{0.in}
%% \setlength{\topmargin}{0.25in}
%
%
%
%\setlength{\oddsidemargin}{0.0in}
%\setlength{\topmargin}{-0.5in}
%\setlength{\headheight}{0.20in}
%\setlength{\headsep}{3ex}
%\setlength{\baselineskip}{2ex}
%\setlength{\textheight}{9in}
%\setlength{\textwidth}{6.4in}
%\renewcommand{\baselinestretch}{1.1}

% Sets margins to 1 in
%\addtolength{\oddsidemargin}{-.5in}%
%\addtolength{\evensidemargin}{-.5in}%
%\addtolength{\textwidth}{1in}%
%\addtolength{\textheight}{1.3in}%
%\addtolength{\topmargin}{-.8in}%

%\setlength{\headheight}{0.20in}
%\setlength{\headsep}{3ex}
%\setlength{\headrulewidth}{0.2pt}
%\setlength{\footrulewidth}{0.15pt}
%\setlength{\parskip}{2.3ex}
% %set to no indentation
%\setlength{\parindent}{0.0in}
%\setlength{\baselineskip}{2ex}
%\setlength{\textheight}{9.in}
%\setlength{\textwidth}{6.5in}

\def \doublespace{\openup 2\jot}
% For double or 1.5 spacing
%\renewcommand{\baselinestretch}{1.5}
\tolerance=500

\def\boxit#1{\vbox{\hrule\hbox{\vrule\kern6pt
\vbox{\kern6pt#1\kern6pt}\kern6pt\vrule}\hrule}}
\renewcommand{\theequation}{\thesection.\arabic{equation}}
% The following for TOC
%\renewcommand{\thepage}{\roman{page}}
% to be followed by this for the main text
\renewcommand{\thepage}{\arabic{page}}


%-----------------------------------------------

%%%%%%%%%%%%%%%%%%%%%%%%%%%%%%%%%%%%%%
%Define any shortcut aliases below

\newtheorem{theo}{Theorem}[section]

\newenvironment{note}{\begin{quote}\emph{Note:\ }}{\end{quote}}
\newenvironment{defn}{
\begin{description}
\item[Definition ]}
{\end{description}}

\newenvironment{ttscript}[1]{%
    \begin{list}{}{%
    \settowidth{\labelwidth}{\texttt{#1}}
    \setlength{\leftmargin}{\labelwidth}
    \addtolength{\leftmargin}{\labelsep}
    \setlength{\parsep}{0.5ex plus0.2ex minus0.2ex}
    \setlength{\itemsep}{0.3ex}
    \renewcommand{\makelabel}[1]{\texttt{##1\hfill}}}}
    {\end{list}}

\newcommand{\bt}{\begin{tabular}}
\newcommand{\et}{\end{tabular}}
\newcommand{\bc}{\begin{center}}
\newcommand{\ec}{\end{center}}
\newcommand{\bi}{\begin{itemize}}
\newcommand{\ei}{\end{itemize}}
\newcommand{\be}{\begin{enumerate}}
\newcommand{\ee}{\end{enumerate}}
\newcommand{\bq}{\begin{quote}}
\newcommand{\eq}{\end{quote}}
\newcommand{\vect}[1]{\mbox{\boldmath $ #1$}}
\newcommand{\avg}[1]{$\overline{#1}$}
\newcommand{\bmp}{\begin{minipage}}
\newcommand{\emp}{\end{minipage}}
\newcommand{\hr}{\u{\hspace{7in}}}
\newcommand{\sr}{\u{\hspace{5in}}}
\newcommand{\chs}{\chi^2}

\newcommand{\labn}[1]{\Large{\textbf{\fbox{Lab #1}}}}


\newcommand{\hd}[1]{\lhead{STAT 417: Lab #1 (Pileggi, W18)}\rhead{Name: \hspace{1.75in}}}
\newcommand{\hdS}[1]{\lhead{STAT 417: Lab #1 (Pileggi, W18)}\rhead{Name: \emph{\textcolor{OrangeRed}{Solutions}}}}
\newcommand{\hdhw}[1]{\lhead{STAT 417: Homework #1 (Pileggi, W18)}\rhead{Name: \hspace{1.75in}}}
\newcommand{\hdhwS}[1]{\lhead{STAT 417: Homework #1 (Pileggi, W18)}\rhead{Name: \emph{\textcolor{OrangeRed}{Solutions}}}}
\newcommand{\bs}{\underline{\hspace{0.5in}}}
\newcommand{\ans}[1]{\emph{\textcolor{OrangeRed}{#1}}}

%\newcommand{\bv}{\footnotesize
%\bmp{.5\textwidth}
%\begin{Verbatim}[frame=single,label=SAS Code,commandchars=\\\{\}],xrightmargin=.5\textwidth}
%
%\newcommand{\ev}{\end{Verbatim}
%\emp
%\normalsize}

\newcommand{\bv}{\begin{code}}
\newcommand{\ev}{\end{code}}

 \newenvironment{code}[1]%
  {\vspace{.1in}\footnotesize\Verbatim[frame=single,label=SAS Code,commandchars=\\\{\},xrightmargin=#1\textwidth,framesep=.2in,labelposition=all]}
  {\endVerbatim\normalsize}

\newenvironment{craw}[2]%
{\vspace{.1in}\footnotesize\Verbatim[frame=single,label=#2,commandchars=\\\{\},xrightmargin=#1\textwidth,framesep=.2in,labelposition=all]}
  {\endVerbatim\normalsize}

\newenvironment{cbox}[1]%
{\vspace{.1in}\footnotesize\Verbatim[frame=single,commandchars=\\\{\},xrightmargin=#1\textwidth,framesep=.2in,labelposition=all]}
  {\endVerbatim\normalsize}

\newcommand{\head}[1]{\large \textbf{#1} \normalsize}

\newcommand{\ttt}[1]{\textbf{\texttt{#1}}}


\newcommand{\bsval}[1]{\underline{\hspace{0.2in}{[#1]}\hspace{0.2in}}}

\newcommand{\ttb}{\textbf}
\newcommand{\tte}{\emph}
\newcommand{\ttu}{\underline}



\newcommand{\jdhr}{\vspace{0.2in}\hrule}


\newcommand{\uspace}[1]{\underline{\hspace{#1}}}

\newenvironment{ident}{\begin{list}{}{}
         \item[]}{\end{list}}

\newenvironment{proposition}{
\begin{description}
\item[Proposition: ]}
{\end{description}}

\newcommand{\bpr}{\begin{proposition}}
\newcommand{\epr}{\end{proposition}}



% \newenvironment{example}
%     {
%         \begin{list}{\textbf{Example:}}
%         {
%         \settowidth{\labelwidth}{}
%         \setlength{\leftmargin}{\labelwidth}
%         }
%     }
%     {\end{list}}


\newenvironment{example}{
\jdhr \vspace{-.17in}\jdhr
\textbf{Example: }}
{}

\newcommand{\bex}{\begin{example}}
\newcommand{\eex}{\end{example}}

\newenvironment{onyourown}{
\jdhr \vspace{-.17in}\jdhr
\textbf{On Your Own: }}
{}

\newcommand{\boy}{\begin{onyourown}}
\newcommand{\eoy}{\end{onyourown}}


%\newenvironment{debug}{
%\jdhr \vspace{-.17in}\jdhr
%\ttb{Debug the Code}
%\fbox{
%\bmp{.95in}
%\includegraphics[height=.35in]{C:/images/bug4.jpg}\includegraphics[height=.35in]{C:/images/buggy8.jpg}
%\emp}
%}
%{\jdhr}

\newenvironment{debug}{
\jdhr \vspace{-.17in}\jdhr
\ttb{Debug the Code: }
\fbox{
\bmp{.95in}
\includegraphics[height=.35in]{C:/images/bug4.jpg}\includegraphics[height=.35in]{C:/images/mushi90.jpg}
\emp}
}
{}


\newcommand{\bbug}{\begin{debug}}
\newcommand{\ebug}{\end{debug}}


\begingroup
  \catcode `_=11
  \gdef\myuscore{_}
  \catcode `~=11
  \gdef\mytilde{~}
  \catcode `\|=0
  \catcode `\\=11
  |gdef|mybs{\}
|endgroup

%Define any shortcut aliases above


%....................................................................
%....................................................................
%....................................................................
%....................................................................
%....................................................................
%....................................................................
%....................................................................
%....................................................................



\usepackage{amssymb}
				




\begin{document}
\hd{2}
%\labn{1}
\vskip10pt

\begin{enumerate}

\item \textbf{[Chocolate Chips Part I]}. Define a time-to-event random variable $T=$ \textit{seconds until a chip completely melts}.
\begin{enumerate}
\item  Under  the assumption that $T$ follows an exponential distribution with scale parameter $\lambda=95$, use Minitab to find:
\begin{enumerate}
\item The probability that a randomly selected chip remains unmelted for longer than 60 seconds.
\item[]
\item[]
\item[]
\item The probability that a randomly selected chip remains unmelted for longer than 90 seconds.
\item[]
\item[]
\item[]
\item The probability that a randomly selected chip will take between 35 and 45 seconds to melt.
\item[]
\item[]
\item[]
\end{enumerate}
\item  Under  the assumption that $T$ follows a log-normal distribution with location parameter $\mu=4.3$ and scale parameter $\sigma=.2$, use Minitab to find:
\begin{enumerate}
\item The probability that a randomly selected chip will take longer than 60 seconds to melt.
\item[]
\item[]
\item[]
\item The probability that a randomly selected chip remains unmelted for longer than 35 seconds.
\item[]
\item[]
\item[]
\item The probability that a randomly selected chip will take between 35 and 45 seconds to melt.
\item[]
\item[]
\item[]
\end{enumerate}
\item  Under  the assumption that $T$ follows a Weibull distribution with shape parameter $\beta=7.1$ and scale parameter $\lambda=77.0$, use Minitab to find:
\begin{enumerate}
\item The probability that a randomly selected chip will take less than 80 seconds to melt.
\item[]
\item[]
\item[]
\item The probability that a randomly selected chip remains unmelted for longer than 70 seconds.
\item[]
\item[]
\item[]
\item The probability that a randomly selected chip will take between 50 and 70 seconds to melt.
\item[]
\item[]
\item[]
\end{enumerate}
\end{enumerate}


\item \textbf{[Chocolate Chips Part II]}. Recall the chocolate chip melting activity.  The times to melt, as well as the censoring status variable have been saved in the Minitab file \texttt{MELT TIMES V2 W2018}.  Use this data set to answer the following questions. Note: the time variable is located in the column \texttt{Seconds}, and  the censoring status variable
is located in the column \texttt{C}. The event time is censored if it's corresponding censoring value is  0.
\begin{enumerate}
\item Under the assumption that $T$ follows a lognormal distribution, use Minitab to construct a graph of the survival function $S(t)$ and estimate the probability that a randomly selected chip remains unmelted for at least 90 seconds (you do not have to sketch the curve).
\item[]
\item[]
\item[]
\item Under the assumption that $T$ follows an exponential distribution, use Minitab to produce a graph of the survival
function $S(t)$ to estimate the probability that a randomly selected chip remains unmelted for at least 90 seconds to melt.
\item[]
\item[]
\item[]
\end{enumerate}
\item \textbf{[Fruit Flies]}. This data set introduced by Partridge and Farquhar (1981), and further
analyzed by Hanley (1983) and by Hanley and Shapiro (1994), was originally analyzed for the purpose of investigating the relationship between increased sexual activity of male fruit flies and their lifetime (in days). The data set is located in the Minitab data set \texttt{Fruitfly}.  The lifetimes of the male fruit flies are located in the column \texttt{Longevity}, and the censoring status values are located in the column \texttt{Censor}.  Let $T$ define the lifetime of a male fruit fly, i.e. the number of days that the fruit fly lived, and assume that $T$ follows an exponential probability distribution. The column \texttt{Partners} contains the number of female mating partners each male fruit fly had during his lifetime and takes the
value 0, 1, or 8. Use Minitab to produce the graphs of the three survival curves corresponding to each number of female partners and answer the following:
\begin{enumerate}
\item Briefly comment on what the curves suggest about the relationship between number of partners and survival for male fruit
flies.
\item[]
\item[]
\item[]
\item The \textbf{median lifetime} is defined to be the value of $T$ such that half the subjects have failed (experienced the event of interest), and half are still alive.  Based on this definition, examine the curves and separately approximate the median lifetimes of the fruit flies who have had 0, 1, and 8 female partners. (You can also find this in the \texttt{Characteristics of Distribution} table in the Minitab output.) Is your answer consistent with the answer provided in part (a)?
\item[]
\item[]
\item[]
\end{enumerate}
\end{enumerate}

\end{document}
