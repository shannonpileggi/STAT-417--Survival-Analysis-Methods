\PassOptionsToPackage{subsection=false}{beamerouterthememiniframes}
\PassOptionsToPackage{dvipsnames,table}{xcolor}
\documentclass[fleqn]{beamer}
\usepackage{graphicx}
\usepackage{multirow}
\usepackage{multicol}
\usepackage{amsmath,amsfonts,amsthm,amsopn}
\usepackage{color, colortbl}
\usepackage{subfig}
\usepackage{wrapfig}
\usepackage{fancybox}
\usepackage{tikz}
\usepackage{fancyhdr}
\usepackage{setspace}
\usepackage{xcolor}
\usepackage{movie15}
\usepackage{pifont}
\usepackage{soul}
\usepackage{fancyvrb,newverbs}
\usepackage{epsfig}
\usepackage{epstopdf}
\fvset{fontsize=\footnotesize}
\RecustomVerbatimEnvironment{verbatim}{Verbatim}{}

%\usepackage{fancybox}

\usetheme{Szeged}
\usecolortheme{default}

%\definecolor{links}{HTML}{2A1B81}
%\definecolor{links}{blue!20}
\hypersetup{colorlinks,linkcolor=,urlcolor=blue!80}

\setbeamertemplate{blocks}[rounded]
\setbeamercolor{block title}{bg=blue!40,fg=black}
\setbeamercolor{block body}{bg=blue!10}


\newenvironment<>{clicker}[1]{%
  \begin{actionenv}#2%
      \def\insertblocktitle{#1}%
      \par%
      \mode<presentation>{%
        \setbeamercolor{block title}{fg=white,bg=magenta}
       \setbeamercolor{block body}{fg=black,bg=magenta!10}
       \setbeamercolor{itemize item}{fg=magenta}
       \setbeamertemplate{itemize item}[triangle]
       \setbeamercolor{enumerate item}{fg=magenta}
     }%
      \usebeamertemplate{block begin}}
    {\par\usebeamertemplate{block end}\end{actionenv}}

%\newcommand{\bmp}{\begin{minipage}}
%\newcommand{\emp}{\end{minipage}}
%\newcommand{\blankcolumn}{\bmp{.05\textwidth}\hspace{0.50in} \emp}

\defbeamertemplate*{footline}{infolines theme}
{
  \leavevmode%
  \hbox{%
  \begin{beamercolorbox}[wd=.333333\paperwidth,ht=2.25ex,dp=1ex,left]{author in head/foot}%
    \usebeamerfont{author in head/foot}~~\insertshortinstitute: \insertshorttitle
  \end{beamercolorbox}%
  \begin{beamercolorbox}[wd=.67\paperwidth,ht=2.25ex,dp=1ex,right]{date in head/foot}%
    \usebeamerfont{date in head/foot}%\insertshortdate{}\hspace*{2em}
    \insertframenumber{} / \inserttotalframenumber\hspace*{2ex}
  \end{beamercolorbox}
  }%
  \vskip0pt%
}

\newcommand{\cmark}{\ding{51}}%
\newcommand{\xmark}{\ding{55}}%
\newcommand{\grp}{\textcolor{magenta}{Group Exercise}}
\newcommand{\grpc}{\textcolor{magenta}{Group Exercise, continued}}
\newcommand{\bsans}[1]{\underline{\hspace{0.2in}\color{blue!80}{#1}\hspace{0.2in}}}

\definecolor{cverbbg}{gray}{0.93}
\newenvironment{cverbatim}
 {\SaveVerbatim{cverb}}
 {\endSaveVerbatim
  \flushleft\fboxrule=0pt\fboxsep=.5em
  \colorbox{cverbbg}{\BUseVerbatim{cverb}}%
  \endflushleft
}
\newenvironment{lcverbatim}
 {\SaveVerbatim{cverb}}
 {\endSaveVerbatim
  \flushleft\fboxrule=0pt\fboxsep=.5em
  \colorbox{cverbbg}{%
    \makebox[\dimexpr\linewidth-2\fboxsep][l]{\BUseVerbatim{cverb}}%
  }
  \endflushleft
}

\newcommand{\bmp}{\begin{minipage}}
\newcommand{\emp}{\end{minipage}}
\newcommand{\blankcolumn}{\bmp{.05\textwidth}\hspace{0.50in} \emp}

 \newenvironment{code}[1]%
  {\vspace{.1in}\footnotesize\Verbatim[frame=single,label=SAS Code,commandchars=\\\{\},xrightmargin=#1\textwidth,framesep=.2in,labelposition=all]}
  {\endVerbatim\normalsize}

 \newenvironment{Rcode}[1]%
  {\vspace{.1in}\footnotesize\Verbatim[frame=single,label=R Code,commandchars=\\\{\},xrightmargin=#1\textwidth,framesep=.2in,labelposition=all]}
  {\endVerbatim\normalsize}

   \newenvironment{RcodeScript}[1]%
  {\vspace{.1in}\scriptsize\Verbatim[frame=single,label=R Code,commandchars=\\\{\},xrightmargin=#1\textwidth,framesep=.2in,labelposition=all]}
  {\endVerbatim\normalsize}

 \newenvironment{RcodeTiny}[1]%
  {\vspace{.1in}\tiny\Verbatim[frame=single,label=R Code,commandchars=\\\{\},xrightmargin=#1\textwidth,framesep=.2in,labelposition=all]}
  {\endVerbatim\normalsize}


   \newenvironment{Rout}[1]%
  {\vspace{.1in}\footnotesize\Verbatim[frame=single,label=R Output,commandchars=\\\{\},xrightmargin=#1\textwidth,framesep=.2in,labelposition=all]}
  {\endVerbatim\normalsize}
  
     \newenvironment{MTout}[1]%
  {\vspace{.1in}\footnotesize\Verbatim[frame=single,label=Minitab Output,commandchars=\\\{\},xrightmargin=#1\textwidth,framesep=.2in,labelposition=all]}
  {\endVerbatim\normalsize}

   \newenvironment{RoutScript}[1]%
  {\vspace{.1in}\scriptsize\Verbatim[frame=single,label=R Output,commandchars=\\\{\},xrightmargin=#1\textwidth,framesep=.2in,labelposition=all]}
  {\endVerbatim\normalsize}

 \newenvironment{RoutTiny}[1]%
  {\vspace{.1in}\tiny\Verbatim[frame=single,label=R Output,commandchars=\\\{\},xrightmargin=#1\textwidth,framesep=.2in,labelposition=all]}
  {\endVerbatim\normalsize}

\newenvironment{craw}[2]%
{\vspace{.1in}\footnotesize\Verbatim[frame=single,label=#2,commandchars=\\\{\},xrightmargin=#1\textwidth,framesep=.2in,labelposition=all]}
  {\endVerbatim\normalsize}


\newenvironment{scriptcraw}[2]%
{\vspace{.1in}\scriptsize \Verbatim[frame=single,label=#2,commandchars=\\\{\},xrightmargin=#1\textwidth,framesep=.2in,labelposition=all]}
  {\endVerbatim\normalsize}

  \newenvironment{tinycraw}[2]%
{\vspace{.1in}\tiny \Verbatim[frame=single,label=#2,commandchars=\\\{\},xrightmargin=#1\textwidth,framesep=.2in,labelposition=all]}
  {\endVerbatim\normalsize}




\title[Set 9]{Cox Regression Models: Inference for $\beta$ and $HR$}
\author[Pileggi]{Shannon Pileggi}

\institute[STAT 417]{STAT 417}

\date{}


\begin{document}

\begin{frame}
\titlepage
\end{frame}

\begin{frame}
\frametitle{OUTLINE\qquad\qquad\qquad} \tableofcontents[hideallsubsections]
\end{frame}


%===========================================================================================================================
\section[Inference for $\beta$ and $HR$]{Inference for $\beta$ and $HR$}
%===========================================================================================================================
\subsection{}

\begin{frame}
\frametitle{Discussion}
Recall the one predictor CR model:
\begin{eqnarray}
h(t|X) = h_0(t)e^{\beta X} \nonumber
\end{eqnarray}
Suppose that the true ratio of hazards for two values of $X$ is $e^{\beta c} = 1$.
\begin{enumerate}
\item  What does this imply about the true of $\beta$?
\item[]
\item[]
\item[]
\item[]
\item  What does this imply about the variable $X$?
\item[]
\item[]
\item[]
\item[]
\end{enumerate}
\end{frame}

\begin{frame}
\frametitle{Hypotheses}
What hypotheses should we use to determine if a single predictor $X$ is associated with the hazard of event occurrence?
\begin{itemize}
\item[$H_0$:]
\item[]
\item[]
\item[]
\item[$H_a$:]
\item[]
\item[]
\item[]
\end{itemize}
\end{frame}

\begin{frame}
\frametitle{Wald test}
The Wald test statistic is given by:
\vskip200pt 
\end{frame}

\begin{frame}[fragile]
\frametitle{NELS college graduation data}
\begin{Rout}{-0.05}
                      coef exp(coef) se(coef)     z Pr(>|z|)
as.factor(Gender)1 0.20862   1.23197  0.08141 2.563   0.0104 *
\end{Rout}
\begin{itemize}
\item State the hypotheses.
\item[]
\item Show the computation of the test statistic.
\item[]
\item[]
\item State a conclusion about the relationship between gender and the hazard of graduating at $\alpha=0.05$ level of significance.
\item[]
\item[]
\item[]
\end{itemize}
\end{frame}

\begin{frame}
\frametitle{CI for $\beta$}
\begin{itemize}
\item The $100(1-\alpha)$\% CI for $\beta$ is given by:
\item[]
\item[]
\item[]
\item[]
\item Interpretation:
\item[]
\item[]
\item[]
\item[]
\item[]
\end{itemize}
\end{frame}

\begin{frame}[fragile]
\frametitle{NELS college graduation data}
\begin{Rout}{-0.05}
                      coef exp(coef) se(coef)     z Pr(>|z|)
as.factor(Gender)1 0.20862   1.23197  0.08141 2.563   0.0104 *
\end{Rout}
Construct and interpret a 95\% CI for $\beta$.
\vskip200pt
\end{frame}

\begin{frame}
\frametitle{CI for $HR$}
\begin{itemize}
\item Categorical predictor (X=0/1): the hazard ratio for subjects with $X=1$ to subjects with $X=0$ is:
\item[]
\item[]
\item Quantitative predictor: the hazard ratio corresponding to a $c$ unit change in $X$ is:
\item[]
\item[]
\item Generally, a $100(1-\alpha)$\% confidence interval for $HR$ is:
\item[]
\item[]
\item[]
\item[]
\item[]
\end{itemize}
\end{frame}

\begin{frame}
\frametitle{Discussion}
\begin{clicker}{What value should we see if a confidence interval for the hazard ratio contains in order to determine if $X$ is associated with hazard?}
\begin{enumerate}[A.]
\item 0
\item 0.5
\item 1
\item none of these
\end{enumerate}
\end{clicker}
\end{frame}

\begin{frame}[fragile]
\frametitle{NELS college graduation data}

\begin{Rout}{-0.05}
                      coef exp(coef) se(coef)     z Pr(>|z|)
as.factor(Gender)1 0.20862   1.23197  0.08141 2.563   0.0104 *
\end{Rout}
Construct and interpret a 95\% confidence interval for the hazard ratio of female to male students.
\vskip200pt
\end{frame}

\begin{frame}[fragile]
\frametitle{NELS college graduation data}
\begin{Rout}{-0.05}
Call:
coxph(formula = Surv(Years, Censor) ~ as.factor(Gender), data = graduate)

  n= 1000, number of events= 614

                      coef exp(coef) se(coef)     z Pr(>|z|)
as.factor(Gender)1 0.20862   1.23197  0.08141 2.563   0.0104 *
---
Signif. codes:  0 ‘***’ 0.001 ‘**’ 0.01 ‘*’ 0.05 ‘.’ 0.1 ‘ ’ 1

                   exp(coef) exp(-coef) lower .95 upper .95
as.factor(Gender)1     1.232     0.8117      1.05     1.445

Concordance= 0.53  (se = 0.012 )
Rsquare= 0.007   (max possible= 0.999 )
Likelihood ratio test= 6.61  on 1 df,   p=0.01016
Wald test            = 6.57  on 1 df,   p=0.01039
Score (logrank) test = 6.59  on 1 df,   p=0.01025
\end{Rout}
\end{frame}

\begin{frame}[fragile]
\frametitle{VALCG lung cancer study}
\begin{Rout}{-0.05}
               coef exp(coef)  se(coef)      z Pr(>|z|)
    karno -0.033424  0.967129  0.005075 -6.586 4.51e-11 ***
\end{Rout}
Construct and interpret a 95\% confidence interval for the hazard ratio for a 10 unit increase in Karnofsky score.
\vskip200pt
\end{frame}

\begin{frame}[fragile]
\frametitle{VALCG lung cancer study}
\begin{Rout}{-0.05}
Call:
coxph(formula = Surv(time, status) ~ karno, data = veteran)

  n= 137, number of events= 128

           coef exp(coef)  se(coef)      z Pr(>|z|)
karno -0.033424  0.967129  0.005075 -6.586 4.51e-11 ***
---
Signif. codes:  0 ‘***’ 0.001 ‘**’ 0.01 ‘*’ 0.05 ‘.’ 0.1 ‘ ’ 1

      exp(coef) exp(-coef) lower .95 upper .95
karno    0.9671      1.034    0.9576    0.9768

Concordance= 0.709  (se = 0.03 )
Rsquare= 0.264   (max possible= 0.999 )
Likelihood ratio test= 42.03  on 1 df,   p=8.983e-11
Wald test            = 43.38  on 1 df,   p=4.513e-11
Score (logrank) test = 45.32  on 1 df,   p=1.674e-11
\end{Rout}
\end{frame}

\begin{frame}
\frametitle{VALCG lung cancer study}
\begin{enumerate}
\item Provide the 95\% confidence interval for both $\beta$ and $HR$ corresponding to a 1 point increase in Karnofsky score.
\item[]
\item[]
\item[]
\item[]
\item[]
\item Construct a 95\% confidence interval for both $\beta$ and $HR$ corresponding to a 10 point increase in Karnofsky score.
\item[]
\item[]
\item[]
\item[]
\item[]
\end{enumerate}
\end{frame}

\begin{frame}
\frametitle{VALCG lung cancer study}
\begin{enumerate}
\setcounter{enumi}{2}
\item How would you respond to a claim that the risk of dying from lung cancer decreases by more than 25\% for a 10 point increase in Karnofsky score?
\item[]
\item[]
\item[]
\item[]
\item[]
\item[]
\item[]
\item[]
\item[]
\end{enumerate}
\end{frame}


\end{document}