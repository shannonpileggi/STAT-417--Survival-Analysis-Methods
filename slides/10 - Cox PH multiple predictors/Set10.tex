\PassOptionsToPackage{subsection=false}{beamerouterthememiniframes}
\PassOptionsToPackage{dvipsnames,table}{xcolor}
\documentclass[fleqn]{beamer}
\usepackage{graphicx}
\usepackage{multirow}
\usepackage{multicol}
\usepackage{amsmath,amsfonts,amsthm,amsopn}
\usepackage{color, colortbl}
\usepackage{subfig}
\usepackage{wrapfig}
\usepackage{fancybox}
\usepackage{tikz}
\usepackage{fancyhdr}
\usepackage{setspace}
\usepackage{xcolor}
\usepackage{movie15}
\usepackage{pifont}
\usepackage{soul}
\usepackage{fancyvrb,newverbs}
\usepackage{epsfig}
\usepackage{epstopdf}
\fvset{fontsize=\footnotesize}
\RecustomVerbatimEnvironment{verbatim}{Verbatim}{}

%\usepackage{fancybox}

\usetheme{Szeged}
\usecolortheme{default}

%\definecolor{links}{HTML}{2A1B81}
%\definecolor{links}{blue!20}
\hypersetup{colorlinks,linkcolor=,urlcolor=blue!80}

\setbeamertemplate{blocks}[rounded]
\setbeamercolor{block title}{bg=blue!40,fg=black}
\setbeamercolor{block body}{bg=blue!10}


\newenvironment<>{clicker}[1]{%
  \begin{actionenv}#2%
      \def\insertblocktitle{#1}%
      \par%
      \mode<presentation>{%
        \setbeamercolor{block title}{fg=white,bg=magenta}
       \setbeamercolor{block body}{fg=black,bg=magenta!10}
       \setbeamercolor{itemize item}{fg=magenta}
       \setbeamertemplate{itemize item}[triangle]
       \setbeamercolor{enumerate item}{fg=magenta}
     }%
      \usebeamertemplate{block begin}}
    {\par\usebeamertemplate{block end}\end{actionenv}}

%\newcommand{\bmp}{\begin{minipage}}
%\newcommand{\emp}{\end{minipage}}
%\newcommand{\blankcolumn}{\bmp{.05\textwidth}\hspace{0.50in} \emp}

\defbeamertemplate*{footline}{infolines theme}
{
  \leavevmode%
  \hbox{%
  \begin{beamercolorbox}[wd=.333333\paperwidth,ht=2.25ex,dp=1ex,left]{author in head/foot}%
    \usebeamerfont{author in head/foot}~~\insertshortinstitute: \insertshorttitle
  \end{beamercolorbox}%
  \begin{beamercolorbox}[wd=.67\paperwidth,ht=2.25ex,dp=1ex,right]{date in head/foot}%
    \usebeamerfont{date in head/foot}%\insertshortdate{}\hspace*{2em}
    \insertframenumber{} / \inserttotalframenumber\hspace*{2ex}
  \end{beamercolorbox}
  }%
  \vskip0pt%
}

\newcommand{\cmark}{\ding{51}}%
\newcommand{\xmark}{\ding{55}}%
\newcommand{\grp}{\textcolor{magenta}{Group Exercise}}
\newcommand{\grpc}{\textcolor{magenta}{Group Exercise, continued}}
\newcommand{\bsans}[1]{\underline{\hspace{0.2in}\color{blue!80}{#1}\hspace{0.2in}}}

\definecolor{cverbbg}{gray}{0.93}
\newenvironment{cverbatim}
 {\SaveVerbatim{cverb}}
 {\endSaveVerbatim
  \flushleft\fboxrule=0pt\fboxsep=.5em
  \colorbox{cverbbg}{\BUseVerbatim{cverb}}%
  \endflushleft
}
\newenvironment{lcverbatim}
 {\SaveVerbatim{cverb}}
 {\endSaveVerbatim
  \flushleft\fboxrule=0pt\fboxsep=.5em
  \colorbox{cverbbg}{%
    \makebox[\dimexpr\linewidth-2\fboxsep][l]{\BUseVerbatim{cverb}}%
  }
  \endflushleft
}

\newcommand{\bmp}{\begin{minipage}}
\newcommand{\emp}{\end{minipage}}
\newcommand{\blankcolumn}{\bmp{.05\textwidth}\hspace{0.50in} \emp}

 \newenvironment{code}[1]%
  {\vspace{.1in}\footnotesize\Verbatim[frame=single,label=SAS Code,commandchars=\\\{\},xrightmargin=#1\textwidth,framesep=.2in,labelposition=all]}
  {\endVerbatim\normalsize}

 \newenvironment{Rcode}[1]%
  {\vspace{.1in}\footnotesize\Verbatim[frame=single,label=R Code,commandchars=\\\{\},xrightmargin=#1\textwidth,framesep=.2in,labelposition=all]}
  {\endVerbatim\normalsize}

   \newenvironment{RcodeScript}[1]%
  {\vspace{.1in}\scriptsize\Verbatim[frame=single,label=R Code,commandchars=\\\{\},xrightmargin=#1\textwidth,framesep=.2in,labelposition=all]}
  {\endVerbatim\normalsize}

 \newenvironment{RcodeTiny}[1]%
  {\vspace{.1in}\tiny\Verbatim[frame=single,label=R Code,commandchars=\\\{\},xrightmargin=#1\textwidth,framesep=.2in,labelposition=all]}
  {\endVerbatim\normalsize}


   \newenvironment{Rout}[1]%
  {\vspace{.1in}\footnotesize\Verbatim[frame=single,label=R Output,commandchars=\\\{\},xrightmargin=#1\textwidth,framesep=.2in,labelposition=all]}
  {\endVerbatim\normalsize}
  
     \newenvironment{MTout}[1]%
  {\vspace{.1in}\footnotesize\Verbatim[frame=single,label=Minitab Output,commandchars=\\\{\},xrightmargin=#1\textwidth,framesep=.2in,labelposition=all]}
  {\endVerbatim\normalsize}

   \newenvironment{RoutScript}[1]%
  {\vspace{.1in}\scriptsize\Verbatim[frame=single,label=R Output,commandchars=\\\{\},xrightmargin=#1\textwidth,framesep=.2in,labelposition=all]}
  {\endVerbatim\normalsize}

 \newenvironment{RoutTiny}[1]%
  {\vspace{.1in}\tiny\Verbatim[frame=single,label=R Output,commandchars=\\\{\},xrightmargin=#1\textwidth,framesep=.2in,labelposition=all]}
  {\endVerbatim\normalsize}

\newenvironment{craw}[2]%
{\vspace{.1in}\footnotesize\Verbatim[frame=single,label=#2,commandchars=\\\{\},xrightmargin=#1\textwidth,framesep=.2in,labelposition=all]}
  {\endVerbatim\normalsize}


\newenvironment{scriptcraw}[2]%
{\vspace{.1in}\scriptsize \Verbatim[frame=single,label=#2,commandchars=\\\{\},xrightmargin=#1\textwidth,framesep=.2in,labelposition=all]}
  {\endVerbatim\normalsize}

  \newenvironment{tinycraw}[2]%
{\vspace{.1in}\tiny \Verbatim[frame=single,label=#2,commandchars=\\\{\},xrightmargin=#1\textwidth,framesep=.2in,labelposition=all]}
  {\endVerbatim\normalsize}




\title[Set 10]{Cox regression models with multiple predictors}
\author[Pileggi]{Shannon Pileggi}

\institute[STAT 417]{STAT 417}

\date{}


\begin{document}

\begin{frame}
\titlepage
\end{frame}

\begin{frame}
\frametitle{OUTLINE\qquad\qquad\qquad} \tableofcontents[hideallsubsections]
\end{frame}


%===========================================================================================================================
\section[Multiple predictors]{Multiple predictors}
%===========================================================================================================================
\subsection{}

\begin{frame}
\frametitle{Cox regression with multiple predictors}
When there are several predictors $X_1,X_2,\ldots,X_p$ that we believe are associated with hazard, then the CR model becomes:
\vskip60pt
Parameters are estimated by maximizing the partial (log) likelihood function and the fitted CR model is given by:
\vskip60pt
\end{frame}

\begin{frame}
\frametitle{Time invariant predictors}
We will assume that the values of each predictor are measured on each individual at the beginning of the study and remain fixed over time - these are called \emph{time invariant predictors}.
\begin{clicker}{For which of the following predictors is it reasonable to assume that they are \emph{time invariant}?  In a study of time to college graduation...}
\begin{enumerate}
\item high school GPA
\item college GPA
\item gender
\item illegal drug use (yes/no)
\item weight
\end{enumerate}
\end{clicker}
\end{frame}

\begin{frame}
\frametitle{Proportionality assumption}
\begin{clicker}{The proportionality assumption of the Cox regression model implies that for \textbf{two sets of values} of predictors, $\{x_1,...,x_p\}$ and $\{x_1^*,...,x_p^*\}$, ...}
\begin{enumerate}
\item the hazard ratio remains constant over time
\item the difference between the hazards remains constant over time
\item the ratio of log hazards remains constant over time
\item the difference between the log hazards remains constant over time
\end{enumerate}
\end{clicker}
\end{frame}

\begin{frame}
\frametitle{Interpretation of ${\beta}_j$'s and $e^{c\beta_j}$'s}
\begin{itemize}
\item[${\beta}_j$]
\item[]
\item[]
\item[]
\item[]
\item[]
\item[]
\item[]
\item[$e^{c\beta_j}$]
\item[]
\item[]
\item[]
\item[]
\item[]
\item[]
\item[]
\end{itemize}
\end{frame}

\begin{frame}
\frametitle{VALCG study with multiple predictors}
Recall the lung cancer study, and consider the predictors:
\begin{itemize}
\item $X_1=$ Karnofsky score (quantitative)
\item $X_2=$ Cancer treatment (0 = standard, 1 = test) (categorical)
\end{itemize}
\begin{enumerate}
\item Write the form of the CR model with the two explanatory variables.
\item[]
\item[]
\item[]
\item Provide an interpretation for $\beta_1$.
\item[]
\item[]
\item[]
\item[]
\item[]
\end{enumerate}
\end{frame}

\begin{frame}
\frametitle{VALCG study with multiple predictors, cont.}
\begin{enumerate}
\setcounter{enumi}{2}
\item Provide an interpretation for $e^{\beta_2}$.
\end{enumerate}
\vskip300pt
\end{frame}

\begin{frame}
\frametitle{VALCG study with multiple predictors: hazard ratios}
Set up the hazard ratios (in terms of $\beta$'s) for:
\begin{enumerate}
\item Patients taking the test treatment to patients taking the standard treatment (fixing Karnofsky score).
\item[]
\item[]
\item[]
\item[]
\item[]
\item A ten point increase in Karnofsky score, fixing the treatment.
\item[]
\item[]
\item[]
\item[]
\item[]
\end{enumerate}
\end{frame}

\begin{frame}
\frametitle{VALCG study with multiple predictors: hazard ratios}
Set up the hazard ratios (in terms of $\beta$'s) for:
\begin{enumerate}
\setcounter{enumi}{2}
\item Patients taking the test treatment whose Karnofsky score is 10 points higher than patients taking the standard treatment.
\item[]
\item[]
\item[]
\end{enumerate}
\vskip200pt
\end{frame}

\begin{frame}[fragile]
\frametitle{VALCG study with multiple predictors: R output}
\begin{Rcode}{-.07}
CR_mod1 <- coxph(Surv(time, status) ~ karno + trt, data = veteran)
summary(CR_mod1)
\end{Rcode}
\vskip5pt
\begin{Rout}{-.07}
           coef exp(coef)  se(coef)      z Pr(>|z|)
karno -0.033954  0.966616  0.005084 -6.679  2.4e-11 ***
trt    0.177322  1.194016  0.183149  0.968    0.333
---
Signif. codes:  0 ‘***’ 0.001 ‘**’ 0.01 ‘*’ 0.05 ‘.’ 0.1 ‘ ’ 1

      exp(coef) exp(-coef) lower .95 upper .95
karno    0.9666     1.0345    0.9570    0.9763
trt      1.1940     0.8375    0.8339    1.7096
\end{Rout}
\end{frame}

\begin{frame}
\frametitle{VALCG study w/ mult. predictors: estimated hazard ratios}
Compute and interpret estimated hazard ratios for:
\begin{enumerate}
\item Patients taking the test treatment to patients taking the standard treatment (fixing Karnofsky score).
\item[]
\item[]
\item[]
\item[]
\item[]
\item A ten point increase in Karnofsky score, fixing the treatment.
\item[]
\item[]
\item[]
\item[]
\item[]
\item[]
\end{enumerate}
\end{frame}

\begin{frame}
\frametitle{VALCG study w/ mult. predictors: estimated hazard ratios}
Compute and interpret estimated hazard ratios for:
\begin{enumerate}
\setcounter{enumi}{2}
\item Patients taking the test treatment whose Karnofsky score is 10 points higher than patients taking the standard treatment.
\item[]
\item[]
\item[]
\end{enumerate}
\vskip200pt
\end{frame}

%===========================================================================================================================
\section[Inference ($\beta_i$)]{Inference ($\beta_i$)}
%===========================================================================================================================
\subsection{}
\begin{frame}
\tableofcontents[currentsection, hideallsubsections]
\end{frame}

\begin{frame}[fragile]
\frametitle{Full R output}
\begin{Rout}{.0}

           coef exp(coef)  se(coef)      z Pr(>|z|)
karno -0.033954  0.966616  0.005084 -6.679  2.4e-11 ***
trt    0.177322  1.194016  0.183149  0.968    0.333
---
Signif. codes:  0 ‘***’ 0.001 ‘**’ 0.01 ‘*’ 0.05 ‘.’ 0.1 ‘ ’ 1

      exp(coef) exp(-coef) lower .95 upper .95
karno    0.9666     1.0345    0.9570    0.9763
trt      1.1940     0.8375    0.8339    1.7096

Concordance= 0.712  (se = 0.03 )
Rsquare= 0.269   (max possible= 0.999 )
Likelihood ratio test= 42.97  on 2 df,   p=4.676e-10
Wald test            = 44.66  on 2 df,   p=2.001e-10
Score (logrank) test = 46.78  on 2 df,   p=6.933e-11
\end{Rout}
\end{frame}

\begin{frame}
\frametitle{Hypothesis test for $\beta_j$}
Test whether the predictor $X_j$ has a significant effect on hazard \textbf{when all other predictors are included in the model} with:
\vskip70pt
The Wald test statistic is:
\vskip70pt
\end{frame}

\begin{frame}[fragile]
\frametitle{VALCG study: Wald tests}
Are treatment and Karnofsky score significant predictors of hazard?
\begin{Rout}{-.07}
           coef exp(coef)  se(coef)      z Pr(>|z|)
karno -0.033954  0.966616  0.005084 -6.679  2.4e-11 ***
trt    0.177322  1.194016  0.183149  0.968    0.333
\end{Rout}
\vskip200pt
\end{frame}

\begin{frame}[fragile]
\frametitle{VALCG study: CI for HR}
Construct and interpret a 95\% confidence interval for the population hazard ratio of patients on the test treatment to those on the standard treatment (for fixed Karnofsky score).
\begin{Rout}{-.07}
           coef exp(coef)  se(coef)      z Pr(>|z|)
karno -0.033954  0.966616  0.005084 -6.679  2.4e-11 ***
trt    0.177322  1.194016  0.183149  0.968    0.333
\end{Rout}
\vskip200pt
\end{frame}

\begin{frame}[fragile]
\frametitle{VALCG study: CI for HR}
Interpret the interval associated with Karnofsky score.
\begin{Rout}{-.07}
      exp(coef) exp(-coef) lower .95 upper .95
karno    0.9666     1.0345    0.9570    0.9763
trt      1.1940     0.8375    0.8339    1.7096
\end{Rout}
\vskip200pt
\end{frame}

%===========================================================================================================================
\section[Inference ($\beta_i + \beta_j$)]{Inference ($\beta_i + \beta_j$)}
%===========================================================================================================================
\subsection{}
\begin{frame}
\tableofcontents[currentsection, hideallsubsections]
\end{frame}

\begin{frame}
\frametitle{Inference for a linear combination of $\beta's$}
The general form of the CI:
\begin{eqnarray}
\exp\left[c\hat{\beta}_j\pm z_{\alpha/2}|c|{SE}(\hat{\beta}_j)\right] \nonumber
\end{eqnarray}
The form of the true hazard ratio for patients taking the test treatment whose Karnofsky score is 10 points
 higher than patients taking the standard treatment:
\begin{eqnarray}
HR = \exp[10\beta_1+\beta_2] \nonumber
\end{eqnarray}
How could we extend the above expression to this linear combination?
\begin{enumerate}
\item Which parts are straightforward?
\item[]
\item[]
\item Which parts need care?
\item[]
\item[]
\end{enumerate}
\end{frame}

\begin{frame}
\frametitle{CI for a linear combination of $\beta's$}
The $100(1-\alpha)\%$ CI for the $HR$ of the general form $e^{(a{\beta}_i+b{\beta}_j)}$ is given by:
\vskip200pt
\end{frame}


\begin{frame}[fragile]
\frametitle{Estimated variance-covariance matrix}
\begin{Rcode}{.4}
CR_mod1\textcolor{OrangeRed}{$var}
\end{Rcode}
\vskip5pt
\begin{Rout}{.4}
              [,1]          [,2]
[1,]  2.584253e-05 -0.0001026675
[2,] -1.026675e-04  0.0335433798
\end{Rout}
\vskip200pt
\end{frame}

\begin{frame}
\frametitle{Estimated variance-covariance matrix}
\begin{itemize}
\item The estimated covariance between $\hat{\beta}_1$ and $\hat{\beta}_2$ is:
\item[]
\item[]
\item[]
% $\widehat{Cov}(\hat{\beta}_1,\hat{\beta}_2)=-.0001026675$
\item The estimated variance of $\hat{\beta}_1$ is:
\item[]
\item[]
\item[]
%, $\hat{V}(\hat{\beta}_1)= .0000258$
\item The estimated variance of $\hat{\beta}_2$ is:
\item[]
\item[]
\item[]
% $\hat{V}(\hat{\beta}_2)= .03354$
\item The standard errors of $\hat{\beta}_1$ and $\hat{\beta}_2$ are:
\item[]
\item[]
\item[]
\end{itemize}
\end{frame}

\begin{frame}
\frametitle{VALCG study: CI for HR for linear combination of $\beta$'s}
Compute the confidence interval for the population hazard ratio for patients taking the test treatment whose Karnofsky score is 10 points higher than patients taking the standard treatment. \\HR = $\exp[10\beta_1+\beta_2]$
\vskip200pt
\end{frame}


%===========================================================================================================================
\section[Overall tests]{Overall tests}
%===========================================================================================================================
\subsection{}
\begin{frame}
\tableofcontents[currentsection, hideallsubsections]
\end{frame}


\begin{frame}[fragile]
\frametitle{Full R output}
\begin{Rout}{.0}

           coef exp(coef)  se(coef)      z Pr(>|z|)
\textcolor{OrangeRed}{karno -0.033954  0.966616  0.005084 -6.679  2.4e-11 ***}
trt    0.177322  1.194016  0.183149  0.968    0.333
---
Signif. codes:  0 ‘***’ 0.001 ‘**’ 0.01 ‘*’ 0.05 ‘.’ 0.1 ‘ ’ 1

      exp(coef) exp(-coef) lower .95 upper .95
karno    0.9666     1.0345    0.9570    0.9763
trt      1.1940     0.8375    0.8339    1.7096

Concordance= 0.712  (se = 0.03 )
Rsquare= 0.269   (max possible= 0.999 )
Likelihood ratio test= 42.97  on 2 df,   p=4.676e-10
\textcolor{OrangeRed}{Wald test            = 44.66  on 2 df,   p=2.001e-10}
Score (logrank) test = 46.78  on 2 df,   p=6.933e-11
\end{Rout}
\end{frame}

\begin{frame}
\frametitle{Overall tests}
What is different about the two highlighted lines?
\vskip200pt
\end{frame}


\begin{frame}
\frametitle{Three tests}
\hspace*{-0.3in}
\begin{minipage}{0.55\textwidth}
\begin{enumerate}
\item Partial Likelihood Ratio Test:
\item[] $\displaystyle \boxed{G_l=2\left[l_p({\boldmath \mbox{$\hat{\beta}$}})-l_p(\boldmath{0})\right]}$
\item[]
\item Wald Test:
\item[] $\displaystyle \boxed{G_W={\boldmath\mbox{$\hat{\beta}$}}^T\mathbf{I}({\boldmath
    \mbox{$\hat{\beta}$}}){\boldmath \mbox{$\hat{\beta}$}}}$
\item[]
\item Score Test:
\item[] $\displaystyle \boxed{G_S=\mathbf{u}^T(\mathbf{0})[\mathbf{I}(\mathbf{0})]^{-1}\mathbf{u}(\mathbf{0})}$
\item[]
\item[] All three test statistics ($G_l$, $G_W$, $G_S$) follow a $\chi^2$-distribution with $p$ degrees of freedom.
\end{enumerate}
\end{minipage}
\blankcolumn
\begin{minipage}{0.40\textwidth}
Details:
\begin{itemize}
\item $l_p({\boldmath\mbox{$\beta$}})$ is the log partial likelihood function
\item $\mathbf{I}({\boldmath \mbox{${\beta}$}})$ is the observed information matrix 
\item $\mathbf{0}=(0,0,\ldots,0)^T$ is a $p$-vector of 0's
\item $\mathbf{u}^T(\mathbf{0})$ is the vector of partial
    derivatives of the log partial likelihood function (also called a
    vector of scores) evaluated at ${\boldmath
    \mbox{${\beta}$}}=\mathbf{0}$
\end{itemize}
\end{minipage}
\end{frame}

\begin{frame}[fragile]
\frametitle{VALCG study: interpret the results}
\begin{Rout}{.0}
Likelihood ratio test= 42.97  on 2 df,   p=4.676e-10
Wald test            = 44.66  on 2 df,   p=2.001e-10
Score (logrank) test = 46.78  on 2 df,   p=6.933e-11
\end{Rout}
\vskip200pt
\end{frame}

\begin{frame}
\frametitle{Partial likelihood ratio test}
\begin{itemize}
\item[$l_p({\boldmath \mbox{$\hat{\beta}$}})$ = ] log partial likelihood function evaluated at the parameter estimates (measures of goodness-of-fit of the CR model to the data when the predictors $X_1,X_2,\ldots,X_p$ are included)
\item[$l_p(\boldmath{0})$ = ] log partial likelihood function evaluated at 0 values (measures of the fit of the \textit{null model}, i.e. a CR model with no predictors and consisting of only the baseline hazard function)
\item[]
\item If $l_p({\boldmath \mbox{$\hat{\beta}$}})$ is ``much larger" than $l_p(\boldmath{0})$, then:
\item[]
\item[]
\item[]
\item The partial likelihood ratio test statistic compares $l_p({\boldmath \mbox{$\hat{\beta}$}})$ to $l_p(\boldmath{0})$:
\item[]
\item[]
\item[]
\end{itemize}
\end{frame}


\begin{frame}[fragile]
\frametitle{Log partial likelihoods in \texttt{R}}
\begin{minipage}{0.43\textwidth}
\begin{Rcode}{.0}
CR_mod1\textcolor{OrangeRed}{$loglik}
\end{Rcode}
\begin{Rout}{.0}
[1] -505.4491 -483.9657
\end{Rout}
\end{minipage}
\blankcolumn
\blankcolumn
\begin{minipage}{0.45\textwidth}
\begin{itemize}
\item[]
\item[]
\item[]
\item[]
\item[$l_p(\boldmath{0})$ = ] left value (-505.4491)
\item[$l_p({\boldmath \mbox{$\hat{\beta}$}})$ = ] right value (-483.9657)
\end{itemize}
\end{minipage}\\
\vskip10pt
Verify that the partial likelihood ratio statistic is $G_l=42.97$.
\vskip200pt
\end{frame}


\begin{frame}[fragile]
\frametitle{$p$-values from $\chi^2$ distribution in \texttt{R}}
\begin{Rcode}{.25}
pchisq(q = 42.97, df = 2, lower.tail = F)
\end{Rcode}
\begin{Rout}{.25}
4.668561e-10
\end{Rout}
\vskip200pt
\end{frame}

%===========================================================================================================================
\section[Dummy variables]{Dummy variables}
%===========================================================================================================================
\subsection{}
\begin{frame}
\tableofcontents[currentsection, hideallsubsections]
\end{frame}


\begin{frame}
\frametitle{Categorical predictors with $>2$ levels}
\begin{itemize}
\item To include a categorical predictor with $k$ levels (i.e. $k$ different possible values) into the CR model, a set of $k$ \textbf{dummy variables}, $D_1,D_2,\ldots,D_k$, must be created to ``represent" the different values that $X$ can take.
\end{itemize}
\vskip200pt
\end{frame}

\begin{frame}
\frametitle{Categorical predictors with $>2$ levels}
Write out the dummy variables required for a CR model for the VALCG lung caner study with the categorical predictor $X=$ \textit{cancer cell type} (\textbf{small cell}, \textbf{squamous}, \textbf{large cell}, and \textbf{adenocarcinoma}).
\vskip200pt
\end{frame}

\begin{frame}
\frametitle{Using dummy variables in the model}
\begin{itemize}
\item Use $k-1$ dummy variables in your model:
\item[]
\item[]
\item[]
\item[]
\item For our lung cancer model, this is:
\item[]
\item[]
\item[]
\item[]
\item The $k^{th}$ dummy variable omitted corresponds to the \textbf{reference cell}.  The results for all other groups are compared relative to the $k^{th}$ value.
\end{itemize}
\end{frame}


\begin{frame}[fragile]
\frametitle{Dummy variables in \texttt{R}}
\begin{itemize}
\item \texttt{R} automatically creates dummy variables for you! If your categorical variable is:
\begin{itemize}
\item \emph{character}: you do not need to use \texttt{as.factor()} (but it won't hurt anything).
\item \emph{numeric}: you \textbf{must} use \texttt{as.factor()} to create the dummy variables.
\end{itemize}
\item[]
\item \texttt{R} also automatically assigns the \textbf{reference} group for you:
\begin{itemize}
\item \emph{character}: first alphabetical value is the reference group
\item \emph{numeric}: first numeric value is the reference group
\end{itemize}
\item[] You can change the reference group in \texttt{R}.
\end{itemize}
\begin{Rcode}{-.05}
CR_mod2 <- coxph(Surv(time, status) ~ celltype, data = veteran)
summary(CR_mod2)
\end{Rcode}
\end{frame}

\begin{frame}[fragile]
\frametitle{VALCG study - CR model with \texttt{celltype}}
Recall that \texttt{celltype} takes on values of: \\ 
\texttt{adeno}, \texttt{large}, \texttt{smallcell}, and \texttt{squamous}. 
\begin{Rout}{-.05}
                     coef exp(coef) se(coef)      z Pr(>|z|)
celltypelarge     -0.9176    0.3995   0.2880 -3.186  0.00144 **
celltypesmallcell -0.1465    0.8638   0.2493 -0.587  0.55687
celltypesquamous  -1.1477    0.3174   0.2929 -3.919  8.9e-05 ***
\end{Rout}
\begin{clicker}{What is the reference group?}
\begin{enumerate}
\item \texttt{adeno}
\item \texttt{large}
\item \texttt{smallcell}
\item \texttt{squamous}
\end{enumerate}
\end{clicker}
\end{frame}

\begin{frame}[fragile]
\frametitle{VALCG study - CR model with \texttt{celltype}}
Write out the estimated Cox regression model.
\begin{Rout}{-.05}
                     coef exp(coef) se(coef)      z Pr(>|z|)
celltypelarge     -0.9176    0.3995   0.2880 -3.186  0.00144 **
celltypesmallcell -0.1465    0.8638   0.2493 -0.587  0.55687
celltypesquamous  -1.1477    0.3174   0.2929 -3.919  8.9e-05 ***
\end{Rout}
\vskip200pt
\end{frame}

\begin{frame}[fragile]
\frametitle{VALCG study - CR model with \texttt{celltype}}
\begin{clicker}{What is the interpretation of $\hat{\beta_1} = -0.9176$?  The \underline{\hspace{0.2in}{1}\hspace{0.2in}} of death for patients with \underline{\hspace{0.2in}{2}\hspace{0.2in}} lung cancer is estimated to be 0.92 \underline{\hspace{0.2in}{3}\hspace{0.2in}} than the \underline{\hspace{0.2in}{4}\hspace{0.2in}} of death for patients with \underline{\hspace{0.2in}{5}\hspace{0.2in}}.}
\begin{enumerate}
\item hazard, log hazard, hazard ratio
\item \texttt{adeno}, \texttt{large}, \texttt{smallcell}, \texttt{squamous}
\item lower, higher, times
\item hazard, log hazard, hazard ratio
\item \texttt{adeno}, \texttt{large}, \texttt{smallcell}, \texttt{squamous}
\end{enumerate}
\end{clicker}
\end{frame}

\begin{frame}[fragile]
\frametitle{VALCG study - CR model with \texttt{celltype}}
\begin{Rout}{-.05}
                     coef exp(coef) se(coef)      z Pr(>|z|)
celltypelarge     -0.9176    0.3995   0.2880 -3.186  0.00144 **
celltypesmallcell -0.1465    0.8638   0.2493 -0.587  0.55687
celltypesquamous  -1.1477    0.3174   0.2929 -3.919  8.9e-05 ***
\end{Rout}
\begin{clicker}{Identify two sets of two groups of cancers that appear to have a similar effect on hazard.  Classify these sets as ``better off'' or ``worse off''.}
\begin{itemize}
\item[]
\item[Set 1:] \hspace{0.2in} \texttt{adeno} \hspace{0.2in} \texttt{large} \hspace{0.2in} \texttt{smallcell} \hspace{0.2in} \texttt{squamous}
\item[]
\item[Set 2:] \hspace{0.2in} \texttt{adeno} \hspace{0.2in} \texttt{large} \hspace{0.2in} \texttt{smallcell} \hspace{0.2in} \texttt{squamous}
\item[]
\end{itemize}
\end{clicker}
\end{frame}

\begin{frame}[fragile]
\frametitle{VALCG study - CR model with \texttt{celltype}}
\begin{Rout}{-.05}
                  exp(coef) exp(-coef) lower .95 upper .95
celltypelarge        \textcolor{OrangeRed}{0.3995}      2.503    0.2272    0.7025
celltypesmallcell    0.8638      1.158    0.5299    1.4080
celltypesquamous     0.3174      3.151    0.1788    0.5634
\end{Rout}
What is the interpretation of 0.3995?
\vskip200pt
\end{frame}

\begin{frame}[fragile]
\frametitle{VALCG study - CR model with \texttt{celltype}}
\begin{Rout}{-.05}
                  exp(coef) exp(-coef) lower .95 upper .95
celltypelarge        0.3995      2.503    0.2272    0.7025
celltypesmallcell    0.8638      1.158    \textcolor{OrangeRed}{0.5299}    \textcolor{OrangeRed}{1.4080}
celltypesquamous     0.3174      3.151    0.1788    0.5634
\end{Rout}
What is the interpretation of the interval 0.5299 - 1.4080?
\vskip200pt
\end{frame}


\begin{frame}[fragile]
\frametitle{VALCG study - CR model with \texttt{celltype}}
\begin{Rout}{-.05}
                     coef exp(coef) se(coef)      z Pr(>|z|)
celltypelarge     -0.9176    0.3995   0.2880 -3.186  0.00144 **
celltypesmallcell -0.1465    0.8638   0.2493 -0.587  0.55687
celltypesquamous  -1.1477    0.3174   0.2929 -3.919  8.9e-05 ***
\end{Rout}
Estimate how many times higher (or percentage points lower) the hazard rate is for patients with small cell cancer than patients with large cell cancer.
\vskip200pt
\end{frame}


\begin{frame}[fragile]
\frametitle{VALCG study - CR model with \texttt{celltype}}
\begin{Rout}{-.05}
                     coef exp(coef) se(coef)      z Pr(>|z|)
celltypelarge     -0.9176    0.3995   0.2880 -3.186  0.00144 **
celltypesmallcell -0.1465    0.8638   0.2493 -0.587  0.55687
celltypesquamous  -1.1477    0.3174   0.2929 -3.919  8.9e-05 ***
\end{Rout}
What is the general approach to construct a CI for the population hazard ratio for patients with small cell cancer relative to patients large cell cancer?
\vskip200pt
\end{frame}

\begin{frame}[fragile]
\frametitle{VALCG study - CR model with \texttt{celltype}}
\begin{Rout}{-.05}
                     coef exp(coef) se(coef)      z Pr(>|z|)
celltypelarge     -0.9176    0.3995   0.2880 -3.186  0.00144 **
celltypesmallcell -0.1465    0.8638   0.2493 -0.587  0.55687
celltypesquamous  -1.1477    0.3174   0.2929 -3.919  8.9e-05 ***
...
Likelihood ratio test= 24.85  on 3 df,   p=1.661e-05
Wald test            = 24.09  on 3 df,   p=2.387e-05
Score (logrank) test = 25.51  on 3 df,   p=1.208e-05
\end{Rout}
Is type of cancer associated with hazard of death?
\vskip200pt
\end{frame}


\end{document} 